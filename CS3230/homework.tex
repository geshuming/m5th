\PassOptionsToPackage{unicode=true}{hyperref} % options for packages loaded elsewhere
\PassOptionsToPackage{hyphens}{url}
\PassOptionsToPackage{dvipsnames,svgnames*,x11names*}{xcolor}
%
\documentclass[british,a4paper,]{article}
\title{CS3230 Homework 1}
\author{Qi Ji}
\date{26th October 2018}
\makeatletter
\let\thetitle\@title
\let\theauthor\@author
\let\thedate\@date
\makeatother

\usepackage{lmodern}
\usepackage{xcolor}
\usepackage{hyperref}
\hypersetup{
            pdftitle={CS3230 Homework 1},
            pdfauthor={Qi Ji},
            colorlinks=true,
            linkcolor=Maroon,
            citecolor=Blue,
            urlcolor=Blue,
            breaklinks=true}
\usepackage{fancyhdr}
\usepackage{amssymb,amsmath,amsthm}
\usepackage{cleveref}
\usepackage{ifxetex,ifluatex}
\usepackage{unicode-math}
\usepackage[vlined,algoruled,linesnumbered]{algorithm2e}

\defaultfontfeatures{Ligatures=TeX,Scale=MatchLowercase}
%\usepackage{xeCJK}
%\setCJKmainfont[]{IPAexMincho}
% use upquote if available, for straight quotes in verbatim environments
\IfFileExists{upquote.sty}{\usepackage{upquote}}{}
% use microtype if available
\IfFileExists{microtype.sty}{%
\usepackage[]{microtype}
\UseMicrotypeSet[protrusion]{basicmath} % disable protrusion for tt fonts
}{}
\IfFileExists{parskip.sty}{%
\usepackage{parskip}
}{% else
\setlength{\parindent}{0pt}
\setlength{\parskip}{6pt plus 2pt minus 1pt}
}
\urlstyle{same}  % don't use monospace font for urls
\newcommand{\passthrough}[1]{#1}
\setlength{\emergencystretch}{3em}  % prevent overfull lines
\providecommand{\tightlist}{%
  \setlength{\itemsep}{0pt}\setlength{\parskip}{0pt}}
\setcounter{secnumdepth}{5}
% Redefines (sub)paragraphs to behave more like sections
\ifx\paragraph\undefined\else
\let\oldparagraph\paragraph
\renewcommand{\paragraph}[1]{\oldparagraph{#1}\mbox{}}
\fi
\ifx\subparagraph\undefined\else
\let\oldsubparagraph\subparagraph
\renewcommand{\subparagraph}[1]{\oldsubparagraph{#1}\mbox{}}
\fi

% set default figure placement to htbp
\makeatletter
\def\fps@figure{htbp}
\makeatother

% load polyglossia as late as possible as it *could* call bidi if RTL lang (e.g. Hebrew or Arabic)
\usepackage{polyglossia}
\setmainlanguage[variant=british]{english}

\theoremstyle{definition}
\newtheorem{statement}{Statement}[subsection]

\newcommand{\set}[1]{\left\{\, #1 \,\right\}}
\newenvironment{soln}{
    \begin{proof}[Solution]
    \renewcommand*{\qedsymbol}{\(\blacksquare\)}
} {
    \end{proof}
}


\pagestyle{fancy}
\lhead{\thetitle}
\chead{}
\rhead{\thedate}
\lfoot{\theauthor}
\cfoot{}
\rfoot{\thepage}
\renewcommand{\footrulewidth}{\headrulewidth}

\begin{document}
\maketitle

\section{K-sorted Array}
\subsection{}
\begin{statement} \label{q1claim}
    Fix \(j\), suppose for any \(i \in \set{1,\dots,j-k-1}\),
    \(B[i]\) contains the \(i\)-th smallest element of \(A\).
    Then the value extracted from the heap will be the \((j-k)\)-th smallest element of \(A\).
\end{statement}

\subsection{}
For any \(i \in \set{1,\dots,n}\), we let \(M(i)\) denote the \(i\)-th smallest element of \(A\).

\begin{proof}
    We first observe that the elements
    \[ X := \set{ A[1], \dots, A[k], \dots, A[\max(j,n)] } \]
    have been added to the heap \(S\) in previous (if any) and current iterations of the \textbf{for} loop.
    Since \(A\) is \(k\)-sorted, \(M(j-k) \in \set{A[1],\dots,A[\max(j-k+k,n)]} = X\).
    By our assumption that \(B[i]\) contains the \(i\)-th smallest element of \(A\) for each \(i\in\set{1,\dots, j-k-1}\).
    We see that the elements
    \[ Y := \set{ M(1), \dots, M(j-k-1) } \]
    have already been extracted from \(S\) in previous iterations.
    As \(A\) contains distinct integers, we see that \(M(j-k)\notin Y\).
    Now we see that the heap \(S\) contains precisely \(X \smallsetminus Y\).
    All elements less than \(M(j-k)\) are not in \(S\), so
    \(M(j-k)\) is minimal in \(S\), and it will be the extracted value.
\end{proof}

\subsection{}
\begin{proof}
    Proceed by induction on \(j-k\).
    Applying \Cref{q1claim} with \(j = k+1\) proves the base case that \(B[1]\) will contain the smallest element of \(A\).
    Similarly, \Cref{q1claim} proves the inductive case.
    This means for every \(i\in\set{1,\dots,n}\), \(B[i] = M(i)\) so in particular, \(B\) contains the elements of \(A\) in sorted order.
\end{proof}

\section{Inversions}

\subsection{}

\begin{soln}
    Given the array \(\left\langle 2,3,8,6,1 \right\rangle\).
    The inversions are
    \( (1,5), (2,5), (3,4),\) \((3,5), (4,5) \).
\end{soln}

\subsection{}

\begin{soln}
    The array given by
    \( \left\langle n, n-1, \dots, 1 \right\rangle \)
    has inversion count
    \( \binom{n}{2} \).
\end{soln}

\subsection{}

\begin{algorithm}
    \caption{Counting inversions with modified mergesort.}
    \DontPrintSemicolon
    \SetKwProg{Sub}{subroutine}{ is}{end}
    \SetKwProg{Fn}{function}{ is}{end}
    \SetKwFunction{Merge}{modified-merge}
    \SetKwFunction{Mergesort}{mergesort}
    \SetKwData{Inversions}{inversions}
    \SetKwData{Left}{left}
    \SetKwData{Mid}{mid}
    \SetKwData{Right}{right}

    \KwData{an array \(A[1,\dots,n]\) containing a permutation of the \(n\) elements}
    \KwResult{the number of inversions in \(A\)}

    \(\Inversions \leftarrow 0\)

    \Sub{\Merge(\Left, \Mid, \Right)}{
        \KwData{indices of start of left subarray, start of right subarray and end of right subarray, where both subarrays sorted}
        \KwResult{two subarrays merged, \Inversions incremented}

        Initialise array $B[\Left,\dots,\Right]$\;
        $i \leftarrow \Left$; $j \leftarrow \Mid$\;
        $k \leftarrow \Left$\;
        \While{$k \leq \Right$}{
            \uIf{$i < \Mid \land \left( j > \Right \lor A[i] < A[j] \right)$}{
                $B[k] \leftarrow A[i]$\;
                $i \leftarrow i + 1$\;
            }
            \Else{
                $B[k] \leftarrow A[j]$\;
                $j \leftarrow j + 1$\;
                $\Inversions \leftarrow \Inversions + (\Mid - i)$\;
            }
            $k \leftarrow k + 1$\;
        }
        copy $B[\Left,\dots,\Right]$ into $A[\Left,\dots,\Right]$\;
    } % end sub

    \Fn{\Mergesort(L, R)}{
        \KwData{start $L$ and end $R$ indices of subarray to mergesort}
        \If{$L = R$}{
            \Return\;
        }
        $M \leftarrow \left\lfloor\frac{L + R}{2}\right\rfloor + 1$\;
        $\Mergesort(L, M-1)$\;
        $\Mergesort(M, R)$\;
        $\Merge(L,M,R)$\;
    }

    \Mergesort(1, n)\;
    output \Inversions\;
\end{algorithm}


\end{document}
