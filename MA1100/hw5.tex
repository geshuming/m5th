\documentclass[12pt]{article}
\usepackage[utf8]{inputenc}
\usepackage[british]{babel}
\usepackage[a4paper,top=2.2cm,bottom=2.4cm,left=2.2cm,right=2.2cm]{geometry}

\usepackage{amsthm, amsmath, amssymb, mathtools}
\mathtoolsset{showonlyrefs}
\usepackage{enumitem}
%\usepackage{braket}
\newcommand{\set}[1]{\left\{\,#1\,\right\}}
\setlength{\parindent}{0pt} % Stop paragraph indentation

\newtheorem{lemma}{Lemma}
\newtheorem*{prop}{Proposition}
\theoremstyle{definition}
\newtheorem{qn}{Q}
\newtheorem*{defn}{Definition}
\numberwithin{equation}{qn}
\numberwithin{lemma}{qn}

\newcommand{\nat}{\mathbb{N}}
\newcommand{\Int}{\mathbb{Z}}
\usepackage[mathscr]{euscript}
\let\euscr\mathscr \let\mathscr\relax% just so we can load this and rsfs
\newcommand{\power}{\euscr{P}}
\newcommand{\idmap}{\mathrm{id}}
\renewcommand{\preceq}{\preccurlyeq}
\renewcommand{\leq}{\leqslant}
\renewcommand{\geq}{\geqslant}

\DeclareMathOperator{\Maps}{Maps}

\begin{document}
12th November 2017\hfill T04
{
    \center\large\textbf{
        MA1100 Fundamental Concepts of Mathematics\\
        AY2017/18 Sem 1\\
        Homework 5
    }
}\\
By: Qi Ji\\
\phantom{By: }A0167793L

% Q1
\begin{qn}
    Let $A$ be a finite set of size $m$ where $m\geq 1$, and let $a$ be an element of $A$.
    Prove that one has $|A\setminus\{a\}| + 1 = m$.
\end{qn}
\begin{proof}
    $A$ is finite, so $\{a\} \subseteq A$ is also finite, by complement principle,
    $|A\setminus\{a\}| + |\{a\}| = |A|$, so $|A\setminus\{a\}| + 1 = m$.
\end{proof}

%\newpage
% Q2
\begin{qn}
    Let $S$ be a finite set and let $f:S\rightarrow S$ be a function.
    Prove that $f$ is injective iff $f$ is surjective.
\end{qn}
\begin{proof}
    Suppose $f: S\rightarrow S$ is injective.
    For any subset $X\subseteq S$, let $f(X) \subseteq S$ be the $X$-image of $f$,
    $$f(X) := \set{y\in S: \exists x\in X.~ f(x) = y}.$$
    Clearly $|f(S)| \leq |S|$ and $|f(S)|$ is finite.
    Since $f$ is injective, by injection principle,
    $|S| \leq |f(S)|$, so $|f(S)| = |S|$.
    By corollary of complement principle, $f(S) = S$ and $f$ is surjective.

    Conversely suppose $f$ is surjective. For any subset $Y\subseteq S$,
    let $f^*(Y) \subseteq S$ denote the $Y$-pre-image of $f$
    $$f^*(Y) := \set{x\in S : f(x) \in Y}.$$
    Clearly $|f^*(S)| \leq |S|$, and since $f$ is surjective,
    for every $y\in S$, $f^*(\{y\})$ is non-empty.
    $$\forall y\in S.~ |f^*(\{y\})| \geq 1$$
    Because $f$ is well-defined, for any distinct pair of elements in $S$,
    the $f$-preimage of their singletons are disjoint.
    $$\forall y_1, y_2\in S.~ y_1\ne y_2 \implies f^*(\{y_1\}) \cap f^*(\{y_2\}) = \emptyset$$
    Because $f$ is totally-defined,
    the union of $f$-preimages of every element in its range will cover the domain $S$,
    so let $|S|$ be $n$, and for $i\in\set{1,2,\dots,n}$, let $y_i$ denote each element in $S$,
    \begin{align*}
        \bigcup_{i=1}^{n} f^*(\{y_i\}) &= S\\
        \left| \bigcup_{i=1}^{n} f^*(\{y\}) \right| &= |S|\\
        \sum_{i=1}^{n} \left| f^*(\{y_i\}) \right| &= n\\
        \shortintertext{for each $y_i$, $f^*(\{y_i\})$ is non-empty}
        1\cdot n \leq \sum_{i=1}^{n} \left| f^*(\{y_i\}) \right| &= n
    \end{align*}
    this means that for each $y_i\in S$, $|f^*(\{y_i\})| = 1$, therefore $f$ is injective.
\end{proof}

% Q3
\begin{qn}
    Let $m,n\in\nat$ be so that $n > m$.
    Prove that there is no injective function $f$ from $\set{1,\dots,n}$ to $\set{1,\dots,m}$.
    (\emph{Pigeonhole Principle})
\end{qn}
\begin{proof}
    First note that $\set{1,\dots,n}\cong \nat_{<n}$ and $\set{1,\dots,m}\cong \nat_{<m}$ are finite,
    \begin{align*}
        n &> m\\
        \left|\set{1,\dots,n}\right| &> \left|\set{1,\dots,m}\right|
    \end{align*}
    Then by (contrapositive of) injection principle,
    there does not exist injective map $f$ from $\set{1,\dots,n}$ to $\set{1,\dots,m}$.
\end{proof}

%\newpage
% Q4
\begin{qn}
    Prove that the function $f:\nat \rightarrow \Int$ defined by
    $f(n) :=
    \begin{cases}
        \frac{n-1}{2};  &\mbox{if } n \text{ is odd},\\
        \frac{-n}{2};  &\mbox{if } n \text{ is even},
    \end{cases}
    $ is bijective.
    (\emph{$\nat$ starts from 1 in this question.})
\end{qn}
\begin{proof}
Define $g:\Int \rightarrow \nat,
    z \mapsto
    \begin{cases}
        -2z; &\mbox{if } z < 0,\\
        2z + 1; &\mbox{if } z \geq 0.
    \end{cases}$

    For any odd $n\in\nat$,
    \[
        (g\circ f)(n) = g\left(\frac{n-1}{2}\right)
                      = 2\left(\frac{n-1}{2}\right) + 1 = n,
    \]
    and for any even $n\in\nat$,
    \[
        (g\circ f)(n) = g\left(\frac{-n}{2}\right)
                      = -2\left(\frac{-n}{2}\right)    = n.
    \]
    So $g\circ f = \idmap_{\nat}$.

    For any $z \in \Int, z < 0$, $-2z > 0$ and is even,
    \[
        (f\circ g)(z) = f(-2z)
                      = \frac{-(-2z)}{2} = z,
    \]
    and when $z \geq 0$, $2z+1 > 0$ and is odd,
    \[
        (f\circ g)(z) = f(2z + 1)
                      = \frac{(2z + 1) - 1}{2} = z.
    \]
\item So $f\circ g = \idmap_{\Int}$. Since $f$ is invertible, $f$ is bijective.
\end{proof}

% Q5
\begin{qn}
    Let $F$ be a finite set and let $I$ be an infinite set.
    Prove that $I\setminus F$ is infinite.
\end{qn}
\begin{proof}
    Without loss of generality, suppose $F\subseteq I$, then $I = F \cup(I\setminus F)$.
    (Otherwise consider the intersection of $F$ and $I$.)
    Suppose for a contradiction $I\setminus F$ is finite,
    since $I\setminus F$ and $F$ are finite and disjoint, by addition principle,
    $$|F| + \left|I\setminus F\right| = \left|F\cup (I\setminus F)\right|$$
    and $F\cup(I\setminus F)$ is also finite.
    But $F\cup(I\setminus F) = I$, so $\idmap_I$ is a bijective map
    from an infinite set to a finite set, a contradiction.
\end{proof}

\newpage
% Q6
\begin{qn}
    Let $S$ be a set.
    Prove that $S$ is countable iff there is an injective function $f:S\rightarrow \nat$.
\end{qn}
\begin{proof}
If $S$ is countable, $S \preceq \nat$
$\iff$ exists injective function $f: S\rightarrow \nat$.
\end{proof}

% Q7
\begin{qn}
    Let $S$ be a non-empty set.
    Prove that $S$ is countable iff there is a surjective map $f:\nat\rightarrow S$.
\end{qn}
\begin{proof}
    If $S$ is countable, $S\preceq \nat \iff$ exists injective map $g:S\rightarrow \nat$
    $\iff$ exists surjective map $f:\nat\rightarrow S$
    (consequence of Axiom of Choice, because $S \ne \emptyset$).
\end{proof}

%\newpage
% Q8
\begin{qn}
    Prove that if $C_1,\dots,C_n$ is countable, then $C_1\times C_2\times\dots\times C_n$ is countable.
\end{qn}
\begin{lemma}
    $\nat\times\nat \cong \nat$. (Another proof in Q11)
\end{lemma}
\begin{proof}
    Consider this visual representation of $\nat\times\nat$
    \[
    \begin{matrix}
        \nat &  0   & 1     & 2     & 3     & \dots    \\
        0   & (0,0) & (0,1) & (0,2) & (0,3) & \dots    \\
        1   & (1,0) & (1,1) & (1,2) & (1,3) & \dots    \\
        2   & (2,0) & (2,1) & (2,2) & (2,3) & \dots    \\
        3   & (3,0) & (3,1) & (3,2) & (3,3) & \dots    \\
        \vdots & \vdots & \vdots & \vdots & \vdots & \ddots    \\
    \end{matrix}
    \]
    Define a bijection $f:\nat\rightarrow \nat\times\nat$
    by diagonally tracing the diagram above, ie
    $f(0) := (0,0), f(1) := (0,1), f(2) := (1,0),
    f(3) := (0,2), f(4) := (1,1), f(5) := (2,0), \dots$.
\end{proof}
\begin{lemma}
    Product of two countable sets is countable.
\end{lemma}
\begin{proof}
    Suppose $C_1, C_2$ are countable sets, $C_1 \preceq \nat$ and $C_2 \preceq \nat$,
    so there exists injective maps
    $f: C_1\rightarrow \nat$ and $g: C_2 \rightarrow \nat$, then
    define $h$ as
    \begin{align*}
        h: C_1\times C_2 &\rightarrow \nat\times\nat,\\
        (c_1, c_2) &\mapsto (f(c_1), g(c_2)).
    \end{align*}
    Suppose $c_1, c_1'\in C_1$ and $c_2, c_2'\in C_2$ such that
    $h(c_1, c_2) = h(c_1', c_2')$, then $(f(c_1), g(c_2)) = (f(c_1'),g(c_2'))$
    which means $f(c_1) = f(c_1')$ and $g(c_2) = g(c_2')$, and because $f$ and $g$ are injective,
    $c_1 = c_1'$ and $c_2 = c_2'$, so $(c_1, c_2) = (c_1', c_2')$ and $h$ is injective.
    This means $C_1\times C_2\preceq \nat\times\nat$, and because $\nat\times\nat\cong\nat$ (from Lemma 8.1), $C_1\times C_2\preceq \nat$.
\end{proof}
\begin{prop}
    Product of finitely many countable sets is countable.
\end{prop}
\begin{proof}
    Product of $2$ countable sets is countable. Now
    suppose the product of $n$ countable sets, $C_1\times C_2\times \dots \times C_n$, is countable,
    $C_1\times C_2\times\dots \times C_n \preceq \nat$, and $C_{n+1}$ is also countable.
    Then by Lemma 8.2, $(C_1\times C_2\times\dots \times C_n)\times C_{n+1} \preceq \nat$.
    Therefore by induction, for any $n\in\nat$, $n\geq 2$, $C_1\times C_2\times\dots\times C_n$ is countable.
\end{proof}

\newpage
% Q9
\begin{qn}
    Let $X$ and $Y$ be any two sets. Suppose $|X| = |Y|$. Show that $|\power(X)| = |\power(Y)|$.
\end{qn}
\begin{proof}
    Suppose $X$ and $Y$ are any two sets where $|X| = |Y|$, then there exists a bijective map
    $f: X \rightarrow Y$.
    For any $C \subseteq X$, the $f$-image of $C$ is a subset of $Y$ where
    \[ f(C) := \set{y\in Y: \exists c\in C.~ f(c) = y} \]
    and because $f$ is bijective, $f(C)$ is uniquely determined by $C$.

    Similarly, for any $D \subseteq Y$, the $f$-preimage of $D$ is a subset of $X$ where
    \[ f^*(D) := \set{x\in X: f(x)\in D} \]
    which is also uniquely determined by $D$ because $f$ is bijective.

    We can define the bijective map $\psi$
    \[ \psi: \power(X) \rightarrow \power(Y),\quad C \mapsto f(C).  \]
    For any $C, C'\in\power(X)$, if $C \ne C'$, then because $f$ is bijective, $f(C)\ne f(C')$,
    so $\psi$ is injective.
    For any $D\in\power(Y)$, because $f$ is surjective, $f^*(D)\subseteq X$,
    so in particular, there exists $C\in\power(X)$ where $f(C) = D$, so $\psi$ is surjective.

    Hence $\psi$ is bijective, and as a result $|\power(X)| = |\power(Y)|$
\end{proof}

\begin{defn}
    For any sets $A$ and $B$, let $\Maps(A,B)$ denote the subset of $A\times B$ defined by
    $$\Maps(A,B) := \left\{\,
        \begin{array}{ll}
            \varphi\in\power(A\times B): & \mbox{$\varphi$ as a relation from $A$ to $B$} \\
                                         & \mbox{is totally defined and well-defined}
        \end{array}
    \,\right\}$$
\end{defn}
% Q10
\begin{qn}
    Let $X$ and $Y$ be any two sets, and consider the set $\Maps(X,Y)$ of all maps from $X$ to $Y$.
    Show that $|\Maps(X,Y)| \leq  |\power(X\times Y)|$.
\end{qn}
\begin{proof}
    Since by definition, $\Maps(X,Y)\subseteq \power(X\times Y)$, define the map
    \begin{align*}
        \Phi: \Maps(X,Y) &\rightarrow \power(X\times Y)     \\
                 \varphi &\mapsto     \varphi
    \end{align*}
    which is almost the identity map, and is clearly injective.
    So $|\Maps(X,Y)| \leq  |\power(X\times Y)|$.
\end{proof}

% Q11
\begin{qn}
    Use the unique prime factorisation property of $\Int$ (fundamental theorem of arithmetic)
    and the Schröder-Bernstein theorem to show that $|\nat| = |\nat\times\nat|$.
\end{qn}
\begin{proof}
    The map $\varphi: \nat\rightarrow\nat\times\nat, n\mapsto (12,n)$ is clearly an injective map
    from $\nat$ to $\nat\times\nat$, so $\nat\preceq \nat\times\nat$.
    Now consider the map $\psi$,
    \begin{align*}
        \psi: \nat\times\nat &\rightarrow \nat,\\
        (a,b) &\mapsto 2^a\cdot3^b
    \end{align*}
    For any $a,b,c,d\in\nat$ where $\psi(a,b) = \psi(c,d)$, $2^a3^b = 2^c3^d$.
    Then by uniqueness of prime factors, $a = c$ and $b = d$, so $(a,b)=(c,d)$, and
    $\psi$ is injective. Therefore $\nat\times\nat \preceq\nat$.

    By Schröder-Bernstein theorem, $|\nat| = |\nat\times\nat|$.
\end{proof}

\newpage
% Q12
\begin{qn}
    Show that $$|\power(\nat)| \leq |\Maps(\nat,\nat)|.$$
    Use this and the above results to deduce that $$|\power(\nat)| = |\Maps(\nat,\nat)|.$$
\end{qn}
\begin{proof}
    For any $S\subseteq \nat$, define $\Psi$,
    \begin{align*}
        \Psi: \power(\nat) &\rightarrow \Maps(\nat,\nat),\\
                         S &\mapsto \lambda_S,           \\
        \text{where } \lambda_S: \nat &\rightarrow\nat,  \\
        n &\mapsto
        \begin{cases}
            1   &\mbox{if } n \in S,\\
            0   &\mbox{otherwise.}
        \end{cases}
    \end{align*}
%    Examples: $\forall n\in\nat.~ \Psi(\emptyset)(n) = 0, t := \Psi(\set{1,2}), t(0) = 0, t(1) = 1, t(2) = 1, t(3) = 0.$
    For any two subsets of $\nat$, $S_1, S_2\in\power(\nat)$, if $S_1 \ne S_2$,
    without loss of generality, $\exists u \in S_1.~ u \not\in S_2$.
    Then $\Psi(S_1)(u) = 1 \ne 0 = \Psi(S_2)(u)$.
    So in particular, $\Psi(S_1) \ne \Psi(S_2).$
    Hence $\Psi$ is injective and $|\power(\nat)| \leq |\Maps(\nat,\nat)|$.

    \begin{flalign}
        \mbox{From Q11,} &&\nat             &\cong \nat\times\nat           &&\\
        \mbox{from Q9,}  &&\power(\nat)     &\cong \power(\nat\times\nat)   &&\\
        \mbox{from Q10,} &&\Maps(\nat,\nat) &\preceq \power(\nat\times\nat) &&\\
        \mbox{therefore} &&\Maps(\nat,\nat) &\preceq \power(\nat)           &&
    \end{flalign}
    Then because
    $\power(\nat) \preceq \Maps(\nat,\nat)$ and $\Maps(\nat,\nat) \preceq \power(\nat)$,
    by Schröder-Bernstein theorem,
    $\power(\nat) \cong \Maps(\nat,\nat)$.
\end{proof}
\end{document}
