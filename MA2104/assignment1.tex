\documentclass{article}
\usepackage[a4paper]{geometry}
\usepackage{sectsty}
\parindent 0pt
\allsectionsfont{\normalfont\sffamily\bfseries}

\author{Qi Ji\\\small T03 \\\small A0167793L}
\title{MA2104 Assignment 1}
\date{28th January 2018}

\usepackage{amsthm, amsmath, amsfonts}

\newtheorem{thm}{Theorem}
\newtheorem*{proposition}{Proposition}
\newtheorem{lemma}[thm]{Lemma}
\newtheorem{corollary}[thm]{Corollary}
\theoremstyle{definition}
\newtheorem{defn}[thm]{Definition}

\usepackage{fancyhdr}
\pagestyle{fancy}
\lhead{Qi Ji -- A0167793L -- T03}
\rhead{MA2104 -- Assignment 1}

\usepackage{mathtools}
\newcommand{\vect}{\overrightarrow}
\DeclarePairedDelimiter\abs{\lvert}{\rvert}
\DeclarePairedDelimiter\coords{\langle}{\rangle}
\DeclarePairedDelimiter\norm{\lVert}{\rVert}
% make unstarred variants scale
\makeatletter
\let\oldabs\abs
\def\abs{\@ifstar{\oldabs}{\oldabs*}}
%
\let\oldnorm\norm
\def\norm{\@ifstar{\oldnorm}{\oldnorm*}}
%
\let\oldcoords\coords
\def\coords{\@ifstar{\oldcoords}{\oldcoords*}}
\makeatother

\DeclareMathOperator{\comp}{comp}
\DeclareMathOperator{\proj}{proj}
\newcommand{\QED}{\tag*{\qedsymbol}}
\begin{document}
\maketitle

\subsection*{Problem 1}
Vector parallel to $L_1$,$\vect{AB} = \coords{1,0,-1}$,
vector parallel to $L_2$,$\vect{CD} = \coords{-2,5,-1}$.
\begin{align*}
    L_1&: \coords{1,1,1} + s\coords{1,0,-1},\\
    L_2&: \coords{3,0,-1} + t\coords{-2,5,-1}.
\end{align*}
Find a vector orthorgonal to both $\vect{AB}$ and $\vect{CD}$,
\[
    \mathbf{n} :=
    \begin{vmatrix}
        \mathbf{i} & \mathbf{j} & \mathbf{k} \\
        1  & 0 & -1  \\
        -2 & 5 & -1
    \end{vmatrix}
    = \coords{5, 3, 5}.
\]
The shortest distance is the absolute value of scalar projection of $\vect{AC} = \coords{2,-1,-2}$ on $\mathbf{n}$.
\begin{align*}
    \abs{ \comp_\mathbf{n} \vect{AC} } &= \abs{ \frac{\vect{AC}\cdot\mathbf{n}}{\norm{\mathbf{n}}} }   \\
    &= \abs{ \frac{-3}{\sqrt{59}} } = \frac{3}{\sqrt{59}}. \QED
\end{align*}

\subsection*{Problem 2}
A vector orthogonal to both $\mathbf{j - k}$ and $\mathbf{i + j}$ will be their cross product,
\[
    \mathbf{n} :=
    \begin{vmatrix}
        \mathbf{i} & \mathbf{j} & \mathbf{k} \\
        0 & 1 & -1  \\
        1 & 1 & 0
    \end{vmatrix}
    = \coords{1,-1,-1}.
\]
Two unit vectors orthogonal to $\mathbf{j - k}$ and $\mathbf{i + j}$ are
\begin{align*}
    \frac{\mathbf{n}}{\norm{\mathbf{n}}} &= \frac{1}{\sqrt{3}}\coords{1,-1,-1}\\
    &= \coords{\frac{1}{\sqrt{3}}, -\frac{1}{\sqrt{3}},-\frac{1}{\sqrt{3}}}, \text{ and}\\
    -\frac{\mathbf{n}}{\norm{\mathbf{n}}} &= \coords{-\frac{1}{\sqrt{3}}, \frac{1}{\sqrt{3}}, \frac{1}{\sqrt{3}}}.   \QED
\end{align*}

\subsection*{Problem 3}
$\vect{PQ} = \coords{2,3,1},
\vect{PS} = \coords{4,2,5},
\vect{QR} = \coords{4,2,5},
\vect{SR} = \coords{2,3,1}$.
Parallelogram is spanned by vectors $\vect{PQ}$ and $\vect{PS}$.
\begin{align*}
    \text{area} = \norm{\vect{PQ} \times \vect{PS}} &=
    \norm{
        \begin{vmatrix}
            \mathbf{i} & \mathbf{j} & \mathbf{k} \\
            2 & 3 & 1  \\
            4 & 2 & 5
        \end{vmatrix}
    }   \\
    &= \norm{\coords{13, -6, -8}} = \sqrt{269}.  \QED
\end{align*}

\subsection*{Problem 4}
$\vect{PQ} = \coords{1,2,1}$, $\vect{PR} = \coords{5, 0, -2}$,
a vector normal to the plane will be their cross product.
\begin{align*}
    \mathbf{n} := \vect{PQ} \times \vect{PR} &=
    \begin{vmatrix}
        \mathbf{i} & \mathbf{j} & \mathbf{k} \\
        1 & 2 & 1   \\
        5 & 0 & -2
    \end{vmatrix}
    \\
    &= \coords{-4, 7, -10}
\end{align*}
Define point $S$ as $\vect{OS} := \vect{OP} + \vect{PQ} + \vect{PR}$.
Area of triangle $PQR$ is half the area of parallelogram $PQRS$, which can be computed as $\norm{\mathbf{n}}$.
Therefore
\begin{align*}
    \text{area of } PQR &= \frac{1}{2} \norm{\mathbf{n}}    \\
    &= \frac{\sqrt{165}}{2} \QED
\end{align*}

\subsection*{Problem 5}
\begin{align*}
    \text{signed volume} &= \mathbf{a}\cdot(\mathbf{b}\times\mathbf{c}) \\
    &= \begin{vmatrix}
        1 & 1 & -1  \\
        1 & -1 & 1  \\
        -1 & 1 & 1
    \end{vmatrix} \\
    &= -1 -1 -1 +1 -1 -1    \\
    &= -4
\end{align*}
Therefore volume of parallelpiped is $4$.\hfill\qedsymbol

\newpage
\subsection*{Problem 6}
To show $\mathbf{v} := (\mathbf{a}\times\mathbf{b}) + (\mathbf{b}\times\mathbf{c}) + (\mathbf{c}\times\mathbf{a})$ is
perpendicular to the plane containing $P, Q, R$,
it suffices to show that the vector is orthogonal to both $\vect{PQ}$ and $\vect{PR}$.

$\vect{PQ} = \mathbf{b} - \mathbf{a}$. To check for orthogonality, compute
$\mathbf{v}\cdot \vect{PQ}$
as follows,
\begin{align*}
    &\phantom{=}\ \left((\mathbf{a}\times\mathbf{b}) + (\mathbf{b}\times\mathbf{c}) + (\mathbf{c}\times\mathbf{a}) \right)
    \cdot (\mathbf{b} - \mathbf{a}) \\
    &= (\mathbf{a}\times\mathbf{b})\cdot(\mathbf{b} - \mathbf{a})
    + (\mathbf{b}\times\mathbf{c}) \cdot(\mathbf{b} - \mathbf{a})
    + (\mathbf{c}\times\mathbf{a}) \cdot(\mathbf{b} - \mathbf{a})   \\
    &= \underbrace{(\mathbf{a}\times\mathbf{b})\cdot\mathbf{b}}_{0}
    - \underbrace{(\mathbf{a}\times\mathbf{b}) \cdot\mathbf{a}}_{0}
    + \underbrace{(\mathbf{b}\times\mathbf{c}) \cdot\mathbf{b}}_{0}
    - (\mathbf{b}\times\mathbf{c}) \cdot\mathbf{a}
    + (\mathbf{c}\times\mathbf{a}) \cdot\mathbf{b}
    - \underbrace{(\mathbf{c}\times\mathbf{a}) \cdot\mathbf{a}}_{0}  \\
    &= - (\mathbf{b}\times\mathbf{c}) \cdot\mathbf{a}
    + (\mathbf{c}\times\mathbf{a}) \cdot\mathbf{b} \tag*{(matrix with duplicate rows has det $0$)} \\
    &= (\mathbf{c}\times\mathbf{b}) \cdot\mathbf{a}
    + (\mathbf{c}\times\mathbf{a}) \cdot\mathbf{b}  \tag*{(by anticommutativity of $\times$)}
\end{align*}
By computing $\mathbf{a}\cdot (\mathbf{c}\times\mathbf{b})$ and $\mathbf{b}\cdot (\mathbf{c}\times\mathbf{a})$,
it is clear that by swapping the 1st row with the 3rd row of the determinant in $\mathbf{a}\cdot (\mathbf{c}\times\mathbf{b})$,
$\mathbf{b}\cdot (\mathbf{c}\times\mathbf{a})$ is obtained.
Thus by property of determinant, $\mathbf{a}\cdot (\mathbf{c}\times\mathbf{b}) = -\mathbf{b}\cdot (\mathbf{c}\times\mathbf{a})$
and we have $\mathbf{v}\cdot \vect{PQ} = 0$, so $\mathbf{v}\perp \vect{PQ}$.

Similarly, we can compute $\mathbf{v} \cdot \vect{PR}$, where $\vect{PR} = (\mathbf{c} - \mathbf{a})$,
\begin{align*}
    &\phantom{=}\ \left((\mathbf{a}\times\mathbf{b}) + (\mathbf{b}\times\mathbf{c}) + (\mathbf{c}\times\mathbf{a}) \right)
    \cdot (\mathbf{c} - \mathbf{a}) \\
    &= (\mathbf{a}\times\mathbf{b})\cdot(\mathbf{c} - \mathbf{a})
    + (\mathbf{b}\times\mathbf{c}) \cdot(\mathbf{c} - \mathbf{a})
    + (\mathbf{c}\times\mathbf{a}) \cdot(\mathbf{c} - \mathbf{a})   \\
    &= (\mathbf{a}\times\mathbf{b})\cdot\mathbf{c}
    - \underbrace{(\mathbf{a}\times\mathbf{b}) \cdot\mathbf{a}}_{0}
    + \underbrace{(\mathbf{b}\times\mathbf{c}) \cdot\mathbf{c}}_{0}
    - (\mathbf{b}\times\mathbf{c}) \cdot\mathbf{a}
    + \underbrace{(\mathbf{c}\times\mathbf{a}) \cdot\mathbf{c}}_{0}
    - \underbrace{(\mathbf{c}\times\mathbf{a}) \cdot\mathbf{a}}_{0}  \\
    &= (\mathbf{a}\times\mathbf{b})\cdot\mathbf{c}
    - (\mathbf{b}\times\mathbf{c}) \cdot\mathbf{a}  \\
    &= (\mathbf{a}\times\mathbf{b})\cdot\mathbf{c}
    + (\mathbf{c}\times\mathbf{b}) \cdot\mathbf{a}
\end{align*}
By a similar computation, we can determine that
$(\mathbf{a}\times\mathbf{b})\cdot\mathbf{c} = - (\mathbf{c}\times\mathbf{b}) \cdot\mathbf{a}$,
implying $\mathbf{v} \cdot \vect{PR} = 0$, so we have $\mathbf{v}\perp \vect{PR}$.
This completes the proof.
\hfill\qedsymbol

\subsection*{Problem 7}
Compute signed volume of parallelpiped spanned by $\mathbf{a}$,$\mathbf{b}$ and $\mathbf{c}$.
\begin{align*}
    \text{signed volume} &= \begin{vmatrix}
        2 & 4  & -8 \\
        3 & -1 & 3  \\
        -5& 11 & -25
    \end{vmatrix}   \\
    &= 50 - 60 -264 +40 -66 +300    \\
    &= 0
\end{align*}
This implies that the parallelpiped spanned by $\mathbf{a}$,$\mathbf{b}$ and $\mathbf{c}$ is in fact a plane,
meaning that the vectors are coplanar.
\hfill\qedsymbol

\end{document}
