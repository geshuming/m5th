%%%%%%%%%%%%%%%%%%%%%%%%%%%%%%%%%%%%%%%%%
% Cheatsheet
% LaTeX Template
% Version 1.0 (12/12/15)
%
% This template has been downloaded from:
% http://www.LaTeXTemplates.com
%
% Original author:
% Michael Müller (https://github.com/cmichi/latex-template-collection) with
% extensive modifications by Vel (vel@LaTeXTemplates.com)
%
% License:
% The MIT License (see included LICENSE file)
%
%%%%%%%%%%%%%%%%%%%%%%%%%%%%%%%%%%%%%%%%%

%----------------------------------------------------------------------------------------
%	PACKAGES AND OTHER DOCUMENT CONFIGURATIONS
%----------------------------------------------------------------------------------------

\documentclass[11pt]{scrartcl} % 11pt font size

\usepackage[utf8]{inputenc} % Required for inputting international characters
%\usepackage[T1]{fontenc} % Output font encoding for international characters

\usepackage[margin=0pt, landscape]{geometry} % Page margins and orientation

\usepackage{graphicx} % Required for including images

\usepackage{amsmath, amssymb, mathtools}
\usepackage{enumitem}
\usepackage{tikz-cd}
\setitemize{noitemsep,topsep=0pt,parsep=0pt,partopsep=0pt}
\setenumerate{noitemsep,topsep=0pt,parsep=0pt,partopsep=0pt}

\setlength{\unitlength}{1mm} % Set the length that numerical units are measured in
\setlength{\parindent}{0pt} % Stop paragraph indentation

\newcommand{\sectiontitle}[1]{\paragraph{#1} \ } % Custom command for subsection titles
\newcommand{\mtx}[1]{\mathbf{#1}}
\newcommand{\vc}[1]{\boldsymbol{#1}}
\newcommand{\mA}{\mathbf{A}}
\newcommand{\norm}[1]{\left\lVert#1\right\rVert}
% transpose
\newcommand{\T}{^\intercal}
\renewcommand{\implies}{\Rightarrow}
%\renewcommand{\iff}{\Leftrightarrow}
\DeclareMathOperator{\rank}{rank}
\DeclareMathOperator{\nullity}{nullity}
\DeclareMathOperator{\Span}{span}
\newcommand{\real}{\mathbb{R}}
\renewcommand{\leq}{\leqslant}
\renewcommand{\geq}{\geqslant}

%----------------------------------------------------------------------------------------

\begin{document}
\pagestyle{empty}
\setlength{\abovedisplayskip}{6pt}
\setlength{\belowdisplayskip}{6pt}
\setlength{\abovedisplayshortskip}{6pt}
\setlength{\belowdisplayshortskip}{6pt}

\begin{picture}(297,210) % Create a container for the page content

%----------------------------------------------------------------------------------------
%	FIRST COLUMN SPECIFICATION
%----------------------------------------------------------------------------------------

\put(10,197){ % Divide the page
\begin{minipage}[t]{85mm} % Create a box to house text

%----------------------------------------------------------------------------------------
%	HEADING ONE
%----------------------------------------------------------------------------------------

\sectiontitle{Matrices}
\hfill\\
$\mtx{A} = (a_{ij})_{m\times p}$ and $\mtx{B} = (b_{ij})_{p\times n}$, then\\
$\mtx{AB} = (ab_{ij})_{m\times n} = \sum\limits_{k=1}^{p} a_{ik}b_{kj}$.\\
    Matrix Multiplication is associative and distributive (left and right) over addition.\\
    For square matrix $\mA$, $\mA^m\mA^n = \mA^{(m+n)}$\\
    (if $\mA$ invertible, also works for negative $m,n$)
\begin{align*}
    \mA^{\intercal\intercal} &= \mA         \\
    (\mA + \mtx{B})\T &= \mA\T + \mtx{B}\T  \\
    (\mtx{AB})\T &= \mtx{B}\T\mA\T          \\
    \shortintertext{for invertible $\mA, \mtx{B}$:}
    \mA\mA^{-1} &= \mtx{I} = \mA^{-1}\mA    \\
    (c\mA)^{-1} &= \frac{1}{c}\mA^{-1}      \\
    (\mtx{AB})^{-1} &= \mtx{B}^{-1}\mA^{-1}
\end{align*}

\sectiontitle{Elementary Row Operations}
\begin{align*}
    \mtx{E}_k\dots\mtx{E}_2\mtx{E}_1\mA &= \mtx{B}   \\
    \mA &= (\mtx{E}_k\dots\mtx{E}_2\mtx{E}_1)^{-1}\mtx{B}   \\
    \mA &= \mtx{E}_1^{-1}\mtx{E}_2^{-1}\dots\mtx{E}_k^{-1}\mtx{B}
\end{align*}
Using row-reduction to find inverse:
\[ (\mA\mid\mtx{I}) \xrightarrow{\ \text{Gauss-Jordan}\ } (\mtx{I}\mid\mA^{-1}) \]
$\mA^{-1} = \mtx{E}_k\dots\mtx{E}_2\mtx{E}_1 \implies \mA = \mtx{E}_1^{-1}\mtx{E}_2^{-1}\dots\mtx{E}_k^{-1}$
\hfill\\
\sectiontitle{Determinants}\\
    $A_{ij} = (-1)^{i+j} \det(\left<\text{cover row $i$ col $j$ in }\mA\right>)$
    \begin{align*}
        \det(\mtx{E}_{add}) &=1\\
        \det(\mtx{E}_{swap}) &= -1\\
        \det(\mtx{E}_{mult}) &= c \\
        \det(\mtx{A}_{\Delta}) &= a_{11}a_{22}\cdots a_{nn}\\
        \det(\mtx{AB}) &= \det(\mtx{A})\det(\mtx{B})    \\
        \det(c\mtx{A}) &= c^n\mtx{A}
    \end{align*}
%----------------------------------------------------------------------------------------

\end{minipage} % End the first column of text
} % End the first division of the page

%----------------------------------------------------------------------------------------
%	SECOND COLUMN SPECIFICATION
%----------------------------------------------------------------------------------------

\put(105,197){ % Divide the page
\begin{minipage}[t]{85mm} % Create a box to house text

%----------------------------------------------------------------------------------------
%	HEADING FOUR
%----------------------------------------------------------------------------------------

$\mA\ \text{adj}(\mA) = \det(\mA)\ \mtx{I} \implies \mA^{-1} = \dfrac{1}{\det(\mA)}\text{adj}(\mA)$
\[
    \text{adj}(\mA) =
    \begin{bmatrix}
            A_{11} & A_{21} & \dots  & A_{n1} \\
            A_{12} & A_{22} & \dots  & A_{n2} \\
            \vdots & \vdots & \ddots & \vdots \\
            A_{1n} & A_{2n} & \dots  & A_{nn}
    \end{bmatrix}
\]
    Cramer's rule: for $\mA \vc{x} = \vc{b}$,
    $$x_n = \frac{\det(\left<\mA:\text{replace $n$-th col with }\vc{b}\right>)}{\det(\mA)}$$

\sectiontitle{Spaces}\\
    Space notation:
    \begin{itemize}
        \item $\{(a,a-b,2b+c) \mid a,b,c\in\real \}$ (explicit)
        \item $\{(x,y,z) \mid x + y + z = 0\}$ (implicit)
    \end{itemize}

\sectiontitle{Spans \& Containment}\\
    Take $U := \{\vc{u}_1,\vc{u}_2,\dots,\vc{u}_k\}$                    \\
    $\Span(U)$ = set of all linear combinations of $U$                  \\
    $\begin{bmatrix} \vc{u}_1 &\vc{u}_2 &\dots &\vc{u}_k &\vc{v} \end{bmatrix}$
        consistent $\implies v\in\Span(U)$.                             \\
    ref $\begin{bmatrix} \vc{u}_1 &\dots &\vc{u}_k\end{bmatrix}$ no zero-row
        $\implies \Span(U) = \real^k$.                                  \\
    Take $V := \{\vc{v}_1,\vc{v}_2,\dots,\vc{v}_k\}$                         \\
    each $\vc{u}_i \in \Span(V) \iff \Span(U)\subseteq\Span(V)$

\sectiontitle{Subspaces, Linear Independence}\\
    Definition of subspace $V$: $\vc{0} \in V$, and                          \\
    $\forall \vc{u},\vc{v} \in V.~ \forall c,d\in\real .~ c\vc{u}+d\vc{v} \in V$  \\
    $U$ is LI means only trivial solution for                           \\
        $c_1\vc{u}_1 + c_2\vc{u}_2 + \dots + c_k\vc{u}_k = \vc{0}$.

\sectiontitle{Basis and Coordinate systems}\\
    A set of vectors $S$ is a basis for vector space $V$ iff
    \begin{itemize}
        \item $S$ is linearly independent
        \item $S$ spans $V$
    \end{itemize}
    Given basis $S$, and $\vc{v} \in V$
    \begin{align*}
        \vc{v} &= c_1\vc{s}_1 + c_2\vc{s}_2 + \dots + c_k\vc{s}_k       \\
        (\vc{v})_S &= (c_1, c_2, \dots, c_k) \in \real^k                \\
    \end{align*}
%----------------------------------------------------------------------------------------

\end{minipage} % End the second column of text
} % End the second division of the page

%----------------------------------------------------------------------------------------
%	THIRD COLUMN SPECIFICATION
%----------------------------------------------------------------------------------------

\put(200,197){ % Divide the page
\begin{minipage}[t]{85mm} % Create a box to house tex
\sectiontitle{Coordinate Systems}\\
    $\forall \vc{u},\vc{v} \in V.~ \vc{u}=\vc{v} \iff (\vc{u})_S=(\vc{v})_S$ (uniq.) \\
    $\forall \vc{u},\vc{v} \in V.c,d\in\real.~ (c\vc{u}+d\vc{v})_S=c(\vc{u})_S+d(\vc{v})_S$\\
    Let $k = |S|$.\\
    $\vc{v}_1,\vc{v}_2,\dots\in V$, are linear independent $\iff$
    $(\vc{v}_1)_S,(\vc{v}_2)_S,\dots\in \real^k$ are linear independent.    \\
    $\Span\{\vc{v}_1,\vc{v}_2,\dots\} = V \iff\\ \Span\{(\vc{v}_1)_S,(\vc{v}_2)_S,\dots\} = \real^k$.\\

\sectiontitle{Dimensions}\\
    $\dim(V) := |S|$ where $S$ is a basis for $V$, is unique.
    \begin{itemize}
        \item $S$ is linearly independent
        \item $\Span(S) = V$
        \item $|S| = \dim(V)$
    \end{itemize}
    2 of above true $\implies$ all true.

\sectiontitle{Transition Matrices}\\
$S$ and $T$ are bases for $V$, for $v\in V$.
    \begin{align*}
        \mtx{P}_{S,T} &=
        \begin{bmatrix}
            [\vc{s}_1]_T & [\vc{s}_2]_T &
            \dots & [\vc{s}_k]_T
        \end{bmatrix}\\
        \shortintertext{is transition matrix from basis $S$ to $T$. ie}
        [\vc{v}]_T &= P_{S,T} [\vc{v}]_S\\
        [\vc{v}]_S &= P_{S,T}^{-1} [\vc{v}]_T
    \end{align*}

\sectiontitle{Rowsp and Range(Colsp)}\\
    Take $\mtx{R} := \text{ref}(\mtx{A})$ is $m\times n$.
    \begin{itemize}
        \item Row operations preserve rowsp.
        \item Rows in $\mtx{R}$ form basis for rowsp.
        \item Pivot columns in $\mtx{R}$ correspond to
            linearly independent columns in $\mtx{A}$
        \item[] \textbf{Note:} row operations preserve linear (in)-dependence of columns
            but could destroy other information like colsp.
    \end{itemize}
    Column space of $\mtx{A} = \{\mtx{A}\vc{u} \mid \vc{u}\in\real^n \}$.\\
    So $\mtx{A}\vc{x} = \vc{b}$ consistent $\iff \vc{b} \in \text{colsp}(\mtx{A})$.

%----------------------------------------------------------------------------------------

\end{minipage} % End the third column of text
} % End the third division of the page
\end{picture} % End the container for the entire page

%----------------------------------------------------------------------------------------
\begin{picture}(297,210) % Create a container for the page content

%----------------------------------------------------------------------------------------
%	FIRST COLUMN SPECIFICATION
%----------------------------------------------------------------------------------------

\put(10,197){ % Divide the page
\begin{minipage}[t]{85mm} % Create a box to house text

%----------------------------------------------------------------------------------------
%	HEADING ONE
%----------------------------------------------------------------------------------------

\sectiontitle{Rank Nullity}
\begin{align*}
    \rank(\mtx{0}) &= 0         \\
    \rank(\mtx{I}_n) &= n       \\
    \rank(\mA) &\leq \min\{m,n\}    \label{rank}\tag{$\ddagger$}\\
    \rank(\mtx{AB}) &\leq \min\{\rank(\mA), \rank(\mtx{B})\}
    \shortintertext{If equality holds in \eqref{rank}, $\mA$ has \textbf{full rank}.}
    \rank(\mtx{A}) &= \rank(\mtx{A}\T)    \\
    \rank(\mtx{A}) + \nullity(\mtx{A}) &= \text{no. columns}
\end{align*}
\sectiontitle{Kernel/Nullspace}
    $$\text{null}(\mtx{A}) := \{\vc{u} \in\real^n \mid \mtx{A}\vc{u} = \vc{0}\}$$
    Suppose $\mtx{A}\vc{v} = \vc{b}$, then general solution of $\mtx{A}\vc{x} = \vc{b}$
    $$\vc{x} \in \{\vc{u} + \vc{v} \mid \vc{u}\in\text{null}(\mtx{A})\}$$

\sectiontitle{Vectors}\\
    Inner product: $\vc{u}\cdot\vc{v} := \mtx{u}\T\mtx{v}$
    \begin{align*}
        \norm{\vc{u}} &= \sqrt{\vc{u}\cdot\vc{u}}  \\
        \norm{c\vc{u}} &= \left|c\right|\norm{\vc{u}}\\
        d(\vc{u},\vc{v}) &= \norm{\vc{u} - \vc{v}}  \\
        \cos \angle(\vc{u},\vc{v}) &= \frac{\vc{u}\cdot\vc{v}}{\norm{\vc{u}}\norm{\vc{v}}}\\
        \norm{\vc{u}+\vc{v}} &\leq \norm{\vc{u}} + \norm{\vc{v}} \tag{$\bigtriangleup$ ineq.}
    \end{align*}

\sectiontitle{Orthogonal}\\
    $\vc{u}\cdot\vc{v} = 0 \iff \vc{u} \perp \vc{v}$\\
    orthogonal $\implies$ linear independence.\\
    orthonormal $:=$ orthogonal $\land$ norm 1.\\
    To check if $S$ is orthogonal basis for $V$:
    \begin{enumerate}[label=(\roman*)]
        \item $S$ is orthogonal
        \item $|S| = \dim(V)$ or $\Span(S) = V$ (ref: dim)
    \end{enumerate}

\sectiontitle{Projections}\\
    Let $S$ be an orthogonal basis for $V$, then $\forall \vc{w}\in\real^n$,
    \[\vc{p} =
        \frac{\vc{w}\cdot\vc{s}_1}{\vc{s}_1\cdot\vc{s}_1}\vc{s}_1 +
        \frac{\vc{w}\cdot\vc{s}_2}{\vc{s}_2\cdot\vc{s}_2}\vc{s}_2 +
        \dots +
        \frac{\vc{w}\cdot\vc{s}_k}{\vc{s}_k\cdot\vc{s}_k}\vc{s}_k
    \] is projection of $\vc{w}$ on $V$. (existence of projections)
\end{minipage} % End the first column of text
} % End the first division of the page

%----------------------------------------------------------------------------------------
%	SECOND COLUMN SPECIFICATION
%----------------------------------------------------------------------------------------

\put(105,197){ % Divide the page
\begin{minipage}[t]{85mm} % Create a box to house text
    Case $\vc{w} \in V$, then $\vc{p} = \vc{w}$.
    For orthonormal basis, simplify expr as denominator becomes 1.

\sectiontitle{Gram-Schmidt Algorithm}\\
    Basis $U := \{ \vc{u}_1, \vc{u}_2,\dots, \vc{u}_k \}$ for $V$.
    \begin{align*}
        \vc{v}_1 &= \vc{u}_1    \\
        \shortintertext{for $i\in\{2,3,\dots,k\}$}
        \vc{v}_i &= \vc{u}_i
        - \sum_{j=1}^{i-1} \frac{\vc{u}_i\cdot\vc{v}_j}{\vc{v}_j\cdot\vc{v}_j}\  \vc{v}_j
    \end{align*}
    $\{\vc{v}_1, \vc{v}_2,\dots,\vc{v}_k\}$ form orthogonal basis for $V$.

%\sectiontitle{Best Approximations}\\
%    Just project the damn vector.
%    Alternatively find least square solution $\vc{u}$, then $\mtx{A}\vc{u} = \vc{p}$.
%    (If they ask for distance just find $\norm{\vc{v} - \vc{p}}$.
%
%\sectiontitle{Least Square Solution}\\
%    Literally solve $\mtx{A}\T \mtx{A}\vc{x} = \mtx{A}\T\vc{b}$.
%
\sectiontitle{Orthogonal matrices}\\
    iff $\mtx{A}\T = \mtx{A}^{-1}$. rows and cols form orthonormal basis for $\real^n$.\\
    Note: Transition matrix between two orthonormal bases is orthogonal.
    So $\mtx{P}_{T, S} = (\mtx{P}_{S, T})^{-1} = (\mtx{P}_{S, T})\T$

\sectiontitle{Eigenvalues}\\
    $\lambda$ is an eigenvalue of $A$ iff
    $\exists \vc{v}\ne \vc{0}.~ \mtx{A}\vc{v} = \lambda\vc{v}$.\\
    Characteristic polynomial is $\det(\mtx{A} - \lambda\mtx{I}) = 0$\\
    Eigenspace $E_\lambda :=$ nullspace of $(\mtx{A} - \lambda\mtx{I})$.\\
    To diagonalise $n\times n$ matrix $\mtx{A}$,
    \begin{enumerate}
        \item Find all distinct $\lambda_1,\lambda_2,\dots,\lambda_k$,
        \item For each $\lambda_i$, find basis $S_{\lambda i}$ for eigenspace $E_{\lambda i}$
        \item[] (If $|S_{\lambda i}| < p_i$ where $p_i$ is power of $(\lambda - \lambda_i)$ in polynomial,
            then not diagonalisable, abort.)
        \item Set $S = \bigcup_{i\in\{1,\dots,k\}} S_{\lambda i} = \{\vc{u}_1,\vc{u}_2,\dots,\vc{u}_n\}$ 
            (union the eigenbases).
        \item[] $\mtx{A}$ is diagonalisable if $|S| = n$.
    \end{enumerate}
    $\mtx{P} = \begin{bmatrix} \vc{u}_1 & \vc{u}_2 & \dots & \vc{u}_n \end{bmatrix}$ st.
        $\mtx{A} = \mtx{PDP}^{-1}$.\\

    For orthogonally diagonalisable (symmetric) matrix $\mtx{A}$,
    Gram-Schmidt and normalise the bases in step 2,
    then $\mtx{A} = \mtx{PDP}\T$.

\end{minipage} % End the second column of text
} % End the second division of the page

%----------------------------------------------------------------------------------------
%	THIRD COLUMN SPECIFICATION
%----------------------------------------------------------------------------------------

\put(200,197){ % Divide the page
\begin{minipage}[t]{85mm} % Create a box to house tex
\sectiontitle{Linear Transformations}\\
    Suppose $T: V \to W$ is a linear transformation and $S = \left\{\,\vc{u}_1, \vc{u}_2, \dots, \vc{u}_n\,\right\}$ is a basis for $V$.\\
    If $T(S)$ is known, and $\mtx{P}_{E,S}$ is the transition matrix from basis $E$ to $S$,\\
    $\mtx{B} = \begin{bmatrix} T(\vc{u}_1) & T(\vc{u}_2) & \cdots & T(\vc{u}_n) \end{bmatrix}$.
    \[
        \begin{tikzcd}[ampersand replacement=\&]
            V \arrow[r, "\mtx{P}_{E,S}"] \arrow[d, "T"] \& \arrow[ld, "\mtx{B}"] [V]_S  \\
            W
        \end{tikzcd}
    \]
    (generalisable).
\hfill\\

\sectiontitle{Corner case matrices}\\
Standard non-diagonalisable matrix: $\begin{bmatrix} a & 1 \\ 0 & a \end{bmatrix}$.\\
$\mtx{A} = \begin{bmatrix}1&1\\0&0\end{bmatrix};
\mtx{B} = \begin{bmatrix}1&1\\-1&-1\end{bmatrix}$ then
$$\mA\mtx{B}=\mtx{0}\text{ but }\mtx{B}\mA\ne \mtx{0}$$
$
\mtx{A} = \begin{bmatrix}1&2\\0&-1\end{bmatrix};
\mtx{B} = \begin{bmatrix}1&3\\0&-1\end{bmatrix}
$ then
$$\mtx{A}^2 = \mtx{B}^2 = \mtx{I}\text{ but }\mtx{A} \not= \mtx{B}, (\mtx{AB})^2 \not= \mtx{I}$$
$
\mtx{A} = \begin{bmatrix}2&1\\3&2\end{bmatrix};
\mtx{B} = \begin{bmatrix}1&-1\\1&1\end{bmatrix}
$ then
$$(\mtx{AB})^k \not= \mtx{A}^k\mtx{B}^k, (\mtx{AB})\T \not= \mtx{A}\T\mtx{B}\T,
(\mtx{AB})^{-1} \not= \mtx{A}^{-1}\mtx{B}^{-1}$$

    \line(1,0){30}
\sectiontitle{Theorem. Invertible Square Matrix}\\
For any square matrix $\mA$:
    \begin{itemize}[label=$\iff$]
        \item[] $\mtx{A}$ is invertible,
        \item $\mA\vc{x} = \vc{0}$ only has trivial solution,
        \item rref$(\mA) = \mtx{I}$,
        \item $\mA = \mtx{E}_k\dots\mtx{E}_2\mtx{E}_1$,
        \item $\det(\mA) \ne 0$,
        \item rows/cols in $\mA$ form basis for $\real^n$,
        \item $\mA$ has full rank,
        \item $0$ is not an eigenvalue of $\mtx{A}$.
    \end{itemize}



\end{minipage} % End the third column of text
} % End the third division of the page
\end{picture} % End the container for the entire page

%----------------------------------------------------------------------------------------

\end{document}
